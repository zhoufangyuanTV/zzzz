\documentclass{ctexbeamer}
\usefonttheme{serif}
\setCJKmainfont[AutoFakeBold]{KaiTi}
\usepackage{tikz}
\renewcommand{\today}{\number\year 年 \number\month 月 \number\day 日}

\title{玩具火车}
\subtitle{我的第二个 \LaTeX 作品}
\author{zhoufangyuanPT}
\date{\today}
\usebackgroundtemplate{%
\tikz\node[opacity=0.16] {\includegraphics[height=\paperheight,width=\paperwidth]{F6.png}};}

\begin{document}
	\begin{frame}
		\titlepage
	\end{frame}

	\begin{frame}{题目描述}
		Alice(爱丽丝)和 Beatrice(比亚特丽斯)建了一个有 $n$ 个点和 $m$ 条有向边的图。

		某些点是充电站。当火车抵达某个充电站时,它就会被充满电。满电火车驶过恰好 $n$ 段轨道后会没电。

		每个点都有一个开关,用来指定火车从该点离开时驶向的出边。

		Alice 和 Beatrice 打算玩游戏。她们已经分完了所有点:每个点要么归 Alice,要么归 Beatrice。游戏开始时,火车从 $s$ 出发,并充满了电。

		在火车首次进入某个点时(包括游戏开始时),该点的拥有者都要确定开关。开关一旦确定就不可更改。

		火车最终会进入某个环。如果火车能够一直行驶下去,Alice 就赢了。否则 Beatrice 就赢了。

		现在给你一个这样的图,并告诉你每个点的拥有者。对于所有点 $s(1\leq s\leq n)$,判断起始点为 $s$ 时,是 Alice 必胜还是 Beatrice 必胜。
	\end{frame}

	\begin{frame}{数据范围和限制}
		\begin{itemize}
			\item $1\leq n\leq 5000$

			\item $n\leq m\leq 20000$

			\item 至少会有一个充电站。

			\item 每个点至少会有一条边以它为起点。

			\item 无重边,可能有自环。
		\end{itemize}

		\begin{description}
			\item[时间限制] 2 sec

			\item[空间限制] 256 MB
		\end{description}
	\end{frame}

	\begin{frame}[t]{你知道吗?}
		\begin{enumerate}
			\item \textcolor{red}{这题的样例没有任何用处。}

			\item 原本的题面中人名分别为 Arezou 和 Borzou,设定上是一对双胞胎兄妹,但不知道为什么被做 PPT 的人改成了 Alice 和 Beatrice。

			\item 毛啸只花了一个小时就获得了 5 分的好成绩。

			\item 所有在比赛时 AC 了本题的人都在当年(2017年)获得了 IOI 金牌。
		\end{enumerate}
	\end{frame}

	\begin{frame}{游戏的本质}
		\begin{figure}
			\includegraphics[width=\textwidth]{F1.png}
		\end{figure}

		火车愉快的行驶着。

		\begin{figure}
			\includegraphics[width=\textwidth]{F2.png}
		\end{figure}

		突然陷入了一个环,如果环中有充电站,那么 Alice 赢,否则 Beatrice 赢。

		由于满电的车可以行驶 $n$ 条边,所以火车的电足以走到环内的第一个充电站,且该充电站足以支持火车一直在环内行驶下去。
	\end{frame}

	\begin{frame}{一个做法}
		\begin{block}{先做一次拓扑排序}
			考虑什么时候 B 必胜。

			此时火车需要进入一个没有充电站的环,那 Alice 就会尽可能地让火车驶向充电站。

			可以从充电站开始做拓扑排序并 DP,把所有无论 Beatrice 如何操作,火车都会行驶到充电站的点找出来。

			\textcolor{red}{若火车行驶到一个不会到充电站的点,那 B 必胜。}
		\end{block}

		\begin{block}{再做一次拓扑排序}
			再考虑到了充电站的点怎么办。
			
			如果到了充电站后,无论 Alice 如何操作,都会驶向一个 B 的必胜点,那么 Alice 还是会输。

			我们就不考虑充电站,从 B 的必胜点开始再做一次拓扑排序并 DP,剩下的点就是 A 的必胜点了。
		\end{block}
	\end{frame}

	\begin{frame}{耶!爆标了}
		时间复杂度:$O(m)$,$m\leq 20000$

		哇,这就出一道阴间加强版。

		\begin{figure}
			\includegraphics[width=\textwidth]{F3.png}
		\end{figure}

		(我也不知道为什么过了一个 Subtask
	\end{frame}

	\begin{frame}{怎么办?}
		\begin{exampleblock}{一个反例}
			\begin{figure}
				\includegraphics[width=\textwidth]{F4.png}
			\end{figure}
		\end{exampleblock}

		左边两个点虽然不会到 B 的必胜点,但是这样它们就到不了充电站了。

		观察数据范围:$n\leq 5000, m\leq 20000$

		懂了!我们需要一个 $O(nm)$ 的做法。
	\end{frame}

	\begin{frame}{蓝字}
		\textcolor{blue}{如果一个充电站会不可避免地驶向 B 的必胜点,那么这个充电站相当于没有。}

		\begin{proof}
			如果一列火车驶入了该充电站,那么它一定会驶向 B 的必胜点。

			驶入 B 的必胜点后,无论 Alice 怎么操作,最后一定会落入一个没有充电站的环。

			充电站只有在环内的时候才有作用,因此这个充电站相当于不存在。
		\end{proof}
	\end{frame}

	\begin{frame}{正解}
		现在我们重新考虑最初的那个做法在干什么。

		\begin{figure}
			\includegraphics[width=\textwidth]{F5.png}
		\end{figure}

		相当于有一条有若干个充电站的链,我们每次把最后一个充电站删去。

		\textcolor{red}{如果删到不能再删,那么剩下的充电站肯定可以相互到达(包括到自身)。}

		我们只需要把最初的做法做 $n$ 次,每次把是 B 的必胜点的充电站设为没有,就可以求出所有 B 的必胜点了。

		时间复杂度:$O(nm)$
	\end{frame}

	\begin{frame}[t]{The End}
		第一次做完整的 PPT,做得不好,还请原谅。

		谢谢。
	\end{frame}
\end{document}